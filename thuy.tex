%% The MIT License (MIT)
%%
%% Copyright (c) 2015 Daniil Belyakov
%%
%% Permission is hereby granted, free of charge, to any person obtaining a copy
%% of this software and associated documentation files (the "Software"), to deal
%% in the Software without restriction, including without limitation the rights
%% to use, copy, modify, merge, publish, distribute, sublicense, and/or sell
%% copies of the Software, and to permit persons to whom the Software is
%% furnished to do so, subject to the following conditions:
%%
%% The above copyright notice and this permission notice shall be included in all
%% copies or substantial portions of the Software.
%%
%% THE SOFTWARE IS PROVIDED "AS IS", WITHOUT WARRANTY OF ANY KIND, EXPRESS OR
%% IMPLIED, INCLUDING BUT NOT LIMITED TO THE WARRANTIES OF MERCHANTABILITY,
%% FITNESS FOR A PARTICULAR PURPOSE AND NONINFRINGEMENT. IN NO EVENT SHALL THE
%% AUTHORS OR COPYRIGHT HOLDERS BE LIABLE FOR ANY CLAIM, DAMAGES OR OTHER
%% LIABILITY, WHETHER IN AN ACTION OF CONTRACT, TORT OR OTHERWISE, ARISING FROM,
%% OUT OF OR IN CONNECTION WITH THE SOFTWARE OR THE USE OR OTHER DEALINGS IN THE
%% SOFTWARE.

% The font could be set to Windows-specific Calibri by using the 'calibri' option
\documentclass[]{SAMPLE_mcdowellcv}

% For mathematical symbols
\usepackage{amsmath}

% Set applicant's personal data for header
\name{Thuy Trinh}
\address{572/39 Au Co Str., Tan Binh Dist., Saigon, Vietnam}
\contacts{
  thuy.ntrinh@gmail.com \linebreak
  \linkimplicit{trinhngocthuy (Twitter)}{https://twitter.com/trinhngocthuy}
}

\begin{document}

  % Print the header
  \makeheader

    \begin{cvsection}
      {Skills}
      \begin{cvsubsection}{}{}
        \begin{itemize}
          \item Languages: Kotlin (proficient), Java (proficient), Swift (basic), C\# (intermediate), C++ (intermediate).
          \item Tech stacks: RxJava, Dagger, Retrofit, OkHttp, Robolectric, Data Binding Library, Fastlane, Travis CI, Git, GitHub.
          \item GitHub: \linkimplicit{https://github.com/thuytrinh}{https://github.com/thuytrinh}.
          \item Engineering blog: \linkimplicit{https://thuytrinh.github.io}{https://thuytrinh.github.io}.
          \item Author of: \linkimplicit{android-collage-views}{https://github.com/thuytrinh/android-collage-views}, \linkimplicit{DateTimeRangePicker}{https://github.com/skedgo/DateTimeRangePicker}, \linkimplicit{RxProperty}{https://github.com/skedgo/RxProperty}.
        \end{itemize}
      \end{cvsubsection}
    \end{cvsection}

      % Print the content
      \begin{cvsection}{Problem Solving}
        \begin{cvsubsection}{HackerRank}
          \highlight{Noticeable contests and ranks}
          \begin{itemize}
            \item \linkimplicit{101 Hack June 2016 • 259/1979 participants}{https://www.hackerrank.com/contests/101hack38/challenges}
            \item \linkimplicit{Zalando CodeSprint • 452/1478 participants}{https://www.hackerrank.com/contests/zalando-codesprint/challenges}
            \item \linkimplicit{SegFault Jun 2015 • 421/1010 participants}{https://www.hackerrank.com/contests/segfault/challenges}
            \item \linkimplicit{The Magic Lines • 532/1290 participants}{https://www.hackerrank.com/contests/magic-lines-may-2015/challenges}
          \end{itemize}
        \end{cvsubsection}

        \begin{cvsubsection}{LeetCode OJ}
          \highlight{\linkimplicit{Solved Question: 29/385. Accepted Submission: 38/75. Acceptance Rate: 50.7\%}{https://leetcode.com/thuytrinh/}}
        \end{cvsubsection}
      \end{cvsection}

      \begin{cvsection}
        {Work Experience}
        \begin{cvsubsection}
          {Software Engineer (Android) at \linkimplicit{SkedGo}{https://skedgo.com/}}
          {November 2013 - March 2017}
          \highlight{Apps: \linkimplicit{TripGo}{https://play.google.com/store/apps/details?id=com.buzzhives.android.tripplanner}, and \linkimplicit{GoOptus}{https://play.google.com/store/apps/details?id=au.com.optus.android.gooptus}}
            \detail{
              Initially participated in a 3-mate team to develop and maintain TripGo, an award winning city trip planner app that helps
    users travel smarter. Became the Android team lead on July 2015. Led a team of 3 Android engineers to successfully
    catch up with iOS team in terms of app feature count and quality.
            }
          \begin{itemize}
            \item Helped migrate the Ant-based project to \keyword{Gradle}-based project at the very initial stage of \keyword{Android Studio}. Configured the project to allow building multi-flavor and multi-build-type APKs. One project but able to build two different apps (TripGo, GoOptus) sharing a common core.
            \item Designed, shipped, and maintained a white-label Android SDK (called TripKit) for clients to access some RESTful
    TripGo APIs. Performed refactoring critical business-logic parts out of legacy mega-view controllers to finally succeed in shipping the SDK.
            \item Refactored lots of \keyword{AsyncTask}-based code by adopting RxJava. Successfully made implementing asynchronous logic
    easy, composable and testable.
            \item Managed to integrate with \keyword{Dagger}, \keyword{Robolectric}, \keyword{AsserJ} and \keyword{Mockito} to facilitate testing. Adopted the Dependency Injection pattern to increase coverage of testable code. Also, wrote lots of unit tests and instrumentation tests accordingly. On top of that, helped train junior teammates to deliver increasingly more testable code. New app releases shipped at almost zero regression.
            \item Debugged and conquered memory leaks with big help from \keyword{LeakCanary}.
          \end{itemize}
        \end{cvsubsection}

        \begin{cvsubsection}
          {Software Engineer (Android) at \linkimplicit{Cogini}{http://www.cogini.com/}}
          {October 2011 - October 2013}
          \highlight{Apps: \linkimplicit{PicCollage}{https://play.google.com/store/apps/details?id=com.cardinalblue.piccollage.google}}
            \detail{
              Joined the Android team to mainly build, and maintain PicCollage from scratch to a 5-million-downloads app. Collaborated directly with Product team from Cardinal Blue on app specifications to analyze and implement new features.
            }
          \begin{itemize}
            \item Reduced crash rate of \keyword{OutOfMemoryError} by applying optimal caching mechanism for fetching, loading and showing bitmaps.
            \item Implemented a photo picker that fetches remote images via \keyword{Bing Search API} and \keyword{Facebook Graph API}.
            \item Built authentication modules via \keyword{Facebook SDK} and \keyword{Twitter4J}.
            \item Utilized \keyword{ActionBarSherlock} to adopt the action bar design pattern.
          \end{itemize}
        \end{cvsubsection}
      \end{cvsection}

\end{document}
